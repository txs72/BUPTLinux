\documentclass[UTF8,12pt,a4paper]{ctexart}
\ctexset{
	section/format += \raggedright
}
%\usepackage[UTF8, heading = true]{ctex}
\newcommand{\HRule}{\rule{\linewidth}{0.5mm}}
\usepackage{listings}
\usepackage{xcolor}
\usepackage{amsmath, amssymb}
\usepackage[hidelinks]{hyperref}


\usepackage{graphicx}
\usepackage{wrapfig}
\usepackage{subcaption}
\usepackage[font=small]{caption}

\usepackage{fancyhdr}
\pagestyle{fancy}
\lhead{
	\includegraphics[width=0.05\textwidth]{./BUPT_LOGO.png}
}
\chead{北京邮电大学}
\rhead{captain@bupt.edu.cn}
\lfoot{}
\cfoot{\thepage}
\rfoot{}
\renewcommand{\headrulewidth}{0.4pt}
\renewcommand{\headwidth}{\textwidth}
\renewcommand{\footrulewidth}{0pt}

\usepackage[top=1in, bottom=1in, left=1.25in, right=1.25in]{geometry}
\setlength{\headheight}{26pt} 
\usepackage{setspace}
\setstretch{1}%单倍行距

\usepackage{fontspec}
% 衬线字体:Linux Libertine
\setmainfont{Linux Libertine O}
% 无衬线字体:Linux Biolinum
\setsansfont{Linux Biolinum O}
% 等宽/打印机字体:Linux Libertine Mono
\setmonofont{Linux Libertine Mono O}

% 衬线字体:Noto Serif CJK SC
\setCJKmainfont{Noto Serif CJK SC}
% 无衬线字体:Noto Sans CJK SC
\setCJKsansfont{Noto Sans CJK SC}
% 等宽字体/打印机字体:Noto Sans Mono CJK SC
\setCJKmonofont{Noto Sans Mono CJK SC}

\usepackage{chemfig}
\usepackage{skak}

\begin{document}
	\begin{center}
		\large
		\textbf{Linux操作系统第五次作业}
		
		臧志强 2014210358
		
		\today
	\end{center}	
	
	\section{有趣的示例}
	\begin{figure}[htbp]
		\centering
		\begin{subfigure}[b]{0.45\textwidth}
			\centering
			\newgame
			
			\showboard
			\caption{国际象棋}
			\label{fig:chess}
		\end{subfigure}
		~
		\begin{subfigure}[b]{0.45\textwidth}
			\centering
			\chemfig{
				NO_2-[:90]*6(=-(-NO_2)=(-CH_3)-(-O_2N)=-)}
			\caption{2,4,6 -- 三硝基甲苯(TNT)}
			\label{fig:TNT}
		\end{subfigure}
		\caption{有趣的示例}
		\label{fig:demoes}
	\end{figure}

	\section{Apocalypse Tank(天启坦克)}
	
	\begin{quote}
		The apocalypse has begun!
	\end{quote}
	The successor to the old Soviet Mammoth tank, the \textit{Apocalypse Assault Tank} (seen in figure \ref{fig:icon}) was a powerful heavy tank used by the USSR during the first and second iterations of the Third World War. It featured twin 120mm cannons, anti-air missiles, very heavy armour, and self-repair abilities, making it a force to be reckoned with on the battlefield.
	\begin{figure}[h]
		\centering
			\begin{subfigure}[c]{0.1\textwidth}
				\includegraphics[width=\textwidth]{RA2_Apocalypse_Tank_Icons.png}
				\newline
				\newline
				\includegraphics[width=\textwidth]{RA2_Apocalypse_Tank_Veteran_Icons.png}
			\end{subfigure}
		~~
			\begin{subfigure}[c]{0.5\textwidth}
				\includegraphics[width=\textwidth]{RA2_Black_Guard_with_anti-aliasing.jpg}
			\end{subfigure}
		\caption{Apocalypse tank}			
		\label{fig:icon}
	\end{figure}	
%	作为老式苏维埃猛犸坦克的继承者,\textbf{天启突击坦克}是苏联在第三次世界大战的两次重演中使用的一种强大的重型坦克。 它装备有一对120mm火炮、对空导弹,同时拥有重装甲和自我修复能力,无疑是战场上的中坚力量。
	
	
%	\begin{figure}
%		\begin{center}
%			\includegraphics[width=0.38\textwidth]{RA2_Apocalypse_Tank.png}\\
%		\end{center}
%		\label{fig:RA2_APOC}
%	\end{figure}

	
%	\chemfig{
%		H_3C-[:72]{\color{blue}N}*5(- 
%		*6(-(={\color{red}O})-
%		{\color{blue}N}(-CH_3)-
%		(={\color{red}O})-
%		{\color{blue}N}(-CH_3)-=)--
%		{\color{blue}N}=-)}
	

\end{document}
